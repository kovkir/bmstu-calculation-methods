\documentclass{bmstu}

\begin{document}

\makereporttitle
{Информатика и системы управления (ИУ)}
{Программное обеспечение ЭВМ и информационные технологии (ИУ7)}
{\textbf{1}}
{Методы вычислений}
{Венгерский метод решения задачи о назначениях}
{9}
{ИУ7-12М}
{К.Э. Ковалец}
{П.А. Власов}


\setcounter{page}{2}
% \renewcommand{\contentsname}{Содержание} 
% \tableofcontents


\chapter{Теоретическая часть}

\textbf{Цель работы:} изучение венгерского метода решения задачи о назначениях.

\textbf{Задание:}
\begin{enumerate}
    \item Реализовать венгерский метод решения задачи о назначениях в виде программы на ЭВМ.
    \item Провести решение задачи с матрицей стоимостей, заданной в индивидуальном варианте, рассмотрев два случая:
    \begin{itemize}
        \item задача о назначениях является задачей минимизации,
        \item задача о назначениях является задачей максимизации.
    \end{itemize}
\end{enumerate}

\section{Cодержательная и математическая постановки задачи о назаначениях}

\textbf{Содержательная постановка:} имеется $n$ работ и $n$ испытателей; стоимость выполнения $i$-ой работы $j$-ым исполнителем составляет $c_{ij} \geqslant 0$ единиц. Требуется распределить все работы между исполнителями так, чтобы
\begin{itemize}
    \item каждый исполнитель выполнял 1 работу;
    \item каждую работу выполнял только 1 исполнитель;
    \item общая стоимость выполнения всех работ была $min$.
\end{itemize}

Введем управляемые переменные:
\begin{align}
    x_{ij} & =
    \begin{cases}
        1, \: \text{если $i$-ую работу выполняет $j$-ый работник}, \\
        0, \: \text{иначе};
    \end{cases} \nonumber \\
    i, j & = \overline{1; n}. 
\end{align}

Из переменных $x_{ij}$, $i,j = \overline{1; n}$, составим 
\begin{equation}
    X = (x_{ij})_{i,j = \overline{1; n}},
\end{equation}
которую назовем матрицей назначений.

Стоимости выполнения работ также записываем в матрицу
\begin{equation}
    C = (c_{ij})_{i,j = \overline{1; n}},
\end{equation}
называемой матрицей стоимостей.

Тогда:
\begin{enumerate}
    \item Стоимость выполнения работ:
        \begin{equation}
            f = \sum_{i=1}^n \sum_{j=1}^n c_{ij} x_{ij}.
        \end{equation}
    \item Условие того, что $i$-ую работу выполнит один работник:
        \begin{equation}
            \sum_{j=1}^n x_{ij} = 1, \: i = \overline{1; n}.
        \end{equation}
    \item Условие того, что $j$-ый работник выполнит одну работу:
        \begin{equation}
            \sum_{i=1}^n x_{ij} = 1, \: j = \overline{1; n}.
        \end{equation}
\end{enumerate}

Таким образом приходим к \textbf{математической постановке}:
\begin{equation}
    \begin{cases}
        f = \sum_{i=1}^n \sum_{j=1}^n c_{ij} x_{ij} \rightarrow min, \\
        \sum_{j=1}^n x_{ij} = 1, \: i = \overline{1; n}, \\
        \sum_{i=1}^n x_{ij} = 1, \: j = \overline{1; n}, \\
        x_{ij} \in \{0, 1\}, \: i,j = \overline{1; n}.
    \end{cases}
\end{equation}

\section{Исходные данные конкретного варианта}

Вариант \textbf{9}:
\begin{equation}
    C = 
    \begin{bmatrix}
        4 & 7 & 1 & 5 & 5 \\
        6 & 8 & 3 & 7 & 6 \\
        6 & 4 & 5 & 7 & 7 \\
        4 & 2 & 3 & 4 & 9 \\
        8 & 1 & 8 & 3 & 8
    \end{bmatrix}
\end{equation}

\section{Краткое описание венгерского метода}

Схема венгерского метода решения задачи о назначениях представлена на рисунках \ref{img:hungarian_algorithm_1}--\ref{img:hungarian_algorithm_2}. 

\imgs{hungarian_algorithm_1}{h!}{0.57}{Схема венгерского метода решения задачи о назначениях (часть 1)}

\imgs{hungarian_algorithm_2}{h!}{0.63}{Схема венгерского метода решения задачи о назначениях (часть 2)}

\chapter{Практическая часть}

\section{Текст программы}

В листинге \ref{lst:lab_01} представлен код программы.

\mylisting[matlab]{lab_01.m}{firstline=1,lastline=358}{Код программы}{lab_01}{}

\clearpage

\section{Результаты расчетов для задач из индивидуального варианта}

В листинге \ref{lst:minimization_debug} представлены расчеты для задачи минимизации.

\mylisting[text]{minimization_debug.txt}{firstline=1,lastline=175}{Задача минимизации}{minimization_debug}{}

В листинге \ref{lst:maximization_debug} представлены расчеты для задачи максимизации.

\mylisting[text]{maximization_debug.txt}{firstline=1,lastline=121}{Задача максимизации}{maximization_debug}{}

\end{document}
