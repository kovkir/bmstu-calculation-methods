\documentclass{bmstu}

\begin{document}

\makereporttitle
{Информатика и системы управления (ИУ)}
{Программное обеспечение ЭВМ и информационные технологии (ИУ7)}
{\textbf{1}}
{Методы вычислений}
{Метод поразрядного поиска}
{6}
{ИУ7-22М}
{К.Э. Ковалец}
{П.А. Власов}


\setcounter{page}{2}
% \renewcommand{\contentsname}{Содержание} 
% \tableofcontents


\chapter{Теоретическая часть}

\textbf{Цель работы:} изучение метода поразрядного поиска для решения задачи одномерной минимизации.

\textbf{Задание:}
\begin{enumerate}
    \item Реализовать метод поразрядного поиска в виде программы на ЭВМ.
    \item Провести решение задачи:
    \begin{equation}
        \begin{cases}
            f(x) \rightarrow min, \\
            x \in [a, b]
        \end{cases}
    \end{equation}
    для данных индивидуального варианта;
    \item организовать вывод на экран графика целевой функции, найденной точки минимума $(x^*, \ f(x^*))$ и последовательности точек $(x_i, \ f(x_i))$, приближающих точку искомого минимума (для последовательности точек следует предусмотреть возможность <<отключения>> вывода ее на экран).
\end{enumerate}

\begin{table}[H]
    \centering
	\caption{Данные индивидуального варианта}
    \label{tbl:task}
	\begin{tabular}{|c|c|c|}
        \hline
        \textbf{№ вар.} & \textbf{Целевая функция $f(x)$} & \textbf{[a, b]} \\ \hline
        6 &
        \begin{minipage}[t]{12cm}\centering 
            \text{ch $(\frac{3x^3 \ + \ 2x^2 \ - \ 4x \ + \ 5}{3}) \ + \ $th $(\frac{x^3 \ - \ 3\sqrt{2}x \ - \ 2}{2x \ + \ \sqrt{2}}) \ - \ {2.5}$}
        \end{minipage} & 
        [0, 1] \\ \hline
    \end{tabular}
\end{table}

\section{Краткое описание метода поразрядного поиска}

\textbf{Метод поразрядного поиска} является усовершенствованнием метода перебора для уменьшения числа обращений к целевой функции.

\textbf{Основная идея:} на начальном этапе, используя сравнительно большой шаг, определяют примерную локализацию точки минимума. Далее в полученной окрестности значение точки минимума уточняют с использованием более мелкого шага (как правило, уменьшенного в 4 раза).

В основе метода лежит известное свойство унимодальных функци: если $x_1 < x_2$, то
\begin{align}
    f(x_1) \leqslant f(x_2) \Rightarrow x^* \in [a, \ x_2], \nonumber \\
    f(x_1) > f(x_2) \Rightarrow x^* \in [x_1, \ b].
\end{align}

Схема рассматриваемого метода представлена на рисунке \ref{img:algorithm}. 

\imgs{algorithm}{h!}{0.6}{Схема алгорима метода поразрядного поиска}


\chapter{Практическая часть}

\begin{table}[H]
    \centering
	\caption{Результаты расчетов для задачи из индивидуального варианта}
    \label{tbl:task}
	\begin{tabular}{|c|c|c|c|c|}
        \hline
        № п/п & $\varepsilon$ & $N$ & $x^*$ & $f(x^*)$ \\ \hline
        1 & 0.01 & 19 & 0.4804687500 & -1.4738794316 \\ \hline
        2 & 0.0001 & 36 & 0.4824218750 & -1.4738932843 \\ \hline
        3 & 0.000001 & 50 & 0.4824180603 & -1.4738932844 \\ \hline
    \end{tabular}
\end{table}

В листинге \ref{lst:lab_01} представлен код программы.

\mylisting[matlab]{lab_01.m}{firstline=1,lastline=74}{Код программы}{lab_01}{}

\end{document}
