\documentclass{bmstu}

\begin{document}

\makereporttitle
{Информатика и системы управления (ИУ)}
{Программное обеспечение ЭВМ и информационные технологии (ИУ7)}
{\textbf{3}}
{Методы вычислений}
{Метод парабол}
{6}
{ИУ7-22М}
{К.Э. Ковалец}
{П.А. Власов}


\setcounter{page}{2}
% \renewcommand{\contentsname}{Содержание} 
% \tableofcontents


\chapter{Теоретическая часть}

\textbf{Цель работы:} изучение метода парабол для решения задачи одномерной минимизации.

\textbf{Задание:}
\begin{enumerate}
    \item Реализовать метод парабол в сочетании с методом золотого сечения в виде программы на ЭВМ.
    \item Провести решение задачи:
    \begin{equation}
        \begin{cases}
            f(x) \rightarrow min, \\
            x \in [a, b]
        \end{cases}
    \end{equation}
    для данных индивидуального варианта;
    \item организовать вывод на экран графика целевой функции, найденной точки минимума $(x^*, \ f(x^*))$ и последовательности точек $(x_i, \ f(x_i))$, приближающих точку искомого минимума (для последовательности точек следует предусмотреть возможность <<отключения>> вывода ее на экран).
\end{enumerate}

\begin{table}[H]
    \centering
	\caption{Данные индивидуального варианта}
    \label{tbl:task}
	\begin{tabular}{|c|c|c|}
        \hline
        \textbf{№ вар.} & \textbf{Целевая функция $f(x)$} & \textbf{[a, b]} \\ \hline
        6 &
        \begin{minipage}[t]{12cm}\centering 
            \text{ch $(\frac{3x^3 \ + \ 2x^2 \ - \ 4x \ + \ 5}{3}) \ + \ $th $(\frac{x^3 \ - \ 3\sqrt{2}x \ - \ 2}{2x \ + \ \sqrt{2}}) \ - \ {2.5}$}
        \end{minipage} & 
        [0, 1] \\ \hline
    \end{tabular}
\end{table}

\section{Краткое описание метода парабол}

Метод парабол является представителем группы методов, основанных на аппроксимации целевой функции некоторой другой функцией, точку минимума которой можно найти аналитически. Эта точка и принимается за очередное приближение искомого минимума целевой функции.

Пусть:
\begin{itemize}
    \item $f(x)$ унимодальна на отрезке $[a, \ b]$;
    \item $f$ достигает минимума во внутренней точке отрезка $[a, \ b]$.
\end{itemize}

Выберем точки $x_1, \ x_2, \ x_3 \in [a, \ b]$ так, чтобы
\begin{equation}
    (*) \ 
    \begin{cases}
        x_1 < x_2 < x_3, \\
        f(x_1) \geqslant f(x_2) \leqslant f(x_3), \ \text{причем по крайней мере одно неравенство} \\
        \text{должно быть строгим}.
    \end{cases}
\end{equation}

Тогда в силу унимодальности функции $f$ $x^* \in [x_1, \ x_3]$.

Аппроксимируем целевую функцию $f(x)$ параболой, проходящей через точки $(x_1, \ f_1)$,  $(x_2, \ f_2)$,  $(x_3, \ f_3)$, где $f_i = f(x_i)$, $i = 1, 2, 3$.
Тогда в силу выбора точек $x_1, \ x_2, \ x_3$ ветви этой параболы будут направлены вверх, а точка $\overline{x}$ ее минимума будет принадлежать отрезку $[x_1, \ x_3]$. За очередное приближение точки $x^*$ принимается точка $\overline{x}$.

Пусть $q(x) = a_0 + a_1(x - x_1) +  a_2(x - x_1)(x - x_2)$ --- уравнение искомой параболы. Тогда можно показать, что 
\begin{equation*}
    \ \ \ \ \ \ \ \ \ \ \ \ \ \ \ \ \ \ a_0 = f_1 \ \text{(не будет использоваться)},
\end{equation*}
\begin{equation}
    (**) \ 
    \begin{cases}
        a_1 = \frac{f_2 - f_1}{x_2 - x_1}, \\
        a_2 = \frac{1}{x_3 - x_2} \left[ \frac{f_3 - f_1}{x_3 - x_1} - \frac{f_2 - f_1}{x_2 - x_1} \right], \\
        \overline{x} \ = \frac{1}{2} \left[ x_1 + x_2 - \frac{a_1}{a_2} \right].
    \end{cases}
\end{equation}

\underline{Замечание}
\begin{enumerate}
    \item О выборе точек $x_1, \ x_2, \ x_3$.
    \begin{enumerate}
        \item На первой итерации обычно достаточно несколько пробных точек. Можно выполнять итерации метода золотого сечения до тех пор, пока для двух пробных точек этого метода и одной из граничных точек очердного отрезка не будут выполнены неравенства $(*)$.
        \item При второй и последующих итерациях на отрезке $[x_1, \ x_3]$ рассматриваются 2 пробные точки $x_2$ и $\overline{x}$, для которых используется метод исключения отрезков. В новом отрезке $[x_1', \ x_3']$ в качестве $x_2'$ выбирается та точка из $x_2$ и $\overline{x}$, которая оказывается внутри.
    \end{enumerate}
    \item На каждой итерации метода парабол, кроме первой, вычисляется 1 значение целевой функции: $\overline{f}$.
\end{enumerate}

\clearpage

Схема рассматриваемого метода представлена на рисунке \ref{img:algorithm}. 

\imgs{algorithm}{h!}{0.6}{Схема алгорима метода поразрядного поиска}


\chapter{Практическая часть}

\begin{table}[H]
    \centering
	\caption{Результаты расчетов для задачи из индивидуального варианта}
    \label{tbl:task}
	\begin{tabular}{|c|c|c|c|c|}
        \hline
        № п/п & $\varepsilon$ & $N$ & $x^*$ & $f(x^*)$ \\ \hline
        1 & 0.01 & 5 & 0.4789477465 & -1.4738494147 \\ \hline
        2 & 0.0001 & 9 & 0.4824113669 & -1.4738932842 \\ \hline
        3 & 0.000001 & 11 & 0.4824179876 & -1.4738932844 \\ \hline
    \end{tabular}
\end{table}

В листинге \ref{lst:lab_03} представлен код программы.

\mylisting[matlab]{lab_03.m}{firstline=1,lastline=181}{Код программы}{lab_03}{}

\end{document}
