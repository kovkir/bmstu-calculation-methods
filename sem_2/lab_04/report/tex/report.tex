\documentclass{bmstu}

\begin{document}

\makereporttitle
{Информатика и системы управления (ИУ)}
{Программное обеспечение ЭВМ и информационные технологии (ИУ7)}
{\textbf{4}}
{Методы вычислений}
{Метод Ньютона}
{6}
{ИУ7-22М}
{К.Э. Ковалец}
{П.А. Власов}


\setcounter{page}{2}
% \renewcommand{\contentsname}{Содержание} 
% \tableofcontents


\chapter{Теоретическая часть}

\textbf{Цель работы:} изучение метода Ньютона для решения задачи одномерной минимизации.

\textbf{Задание:}
\begin{enumerate}
    \item Реализовать модифицированный метод Ньютона с конечно-разностной аппроксимацией производных в виде программы на ЭВМ.
    \item Провести решение задачи:
    \begin{equation}
        \begin{cases}
            f(x) \rightarrow min, \\
            x \in [a, b]
        \end{cases}
    \end{equation}
    для данных индивидуального варианта;
    \item организовать вывод на экран графика целевой функции, найденной точки минимума $(x^*, \ f(x^*))$ и последовательности точек $(x_i, \ f(x_i))$, приближающих точку искомого минимума (для последовательности точек следует предусмотреть возможность <<отключения>> вывода ее на экран);
    \item провести решение задачи с использованием стандартной функции \texttt{fminbnd} пакета $MatLAB$.
\end{enumerate}

\begin{table}[H]
    \centering
	\caption{Данные индивидуального варианта}
    \label{tbl:task}
	\begin{tabular}{|c|c|c|}
        \hline
        \textbf{№ вар.} & \textbf{Целевая функция $f(x)$} & \textbf{[a, b]} \\ \hline
        6 &
        \begin{minipage}[t]{12cm}\centering 
            \text{ch $(\frac{3x^3 \ + \ 2x^2 \ - \ 4x \ + \ 5}{3}) \ + \ $th $(\frac{x^3 \ - \ 3\sqrt{2}x \ - \ 2}{2x \ + \ \sqrt{2}}) \ - \ {2.5}$}
        \end{minipage} & 
        [0, 1] \\ \hline
    \end{tabular}
\end{table}

\clearpage
\section{Краткое описание метода Ньютона}

Для приближения корня $x^*$ используется точка $\overline{x}$ пересечения касательной к графику $g(x)$ в точке текущего приближения с осью $Ox$.

Пусть $\overline{x}$ --- текущее приближение точки $x^*$. Уравнение касательной к графику функции $g$ в точке $(\overline{x}, \ g(\overline{x}))$:
\begin{equation}
    y \ - \ \overline{y} \ = \ g'(\overline{x})(x \ - \ \overline{x}).
\end{equation}

Пересечение с $OX$:
\begin{equation}
    \overline{x} \ = \ \overline{x}' \ - \ \frac{g(\overline{x}')}{g'(\overline{x}')},
\end{equation}
где $\overline{x}'$ --- приближение $x^*$ с предыдущего шага, $\overline{x}$ --- приближение $x^*$ с текущего шага.

Условие окончанния итераций:
\begin{equation}
    |\overline{x} \ - \ \overline{x}'| \ < \ \varepsilon \ \text{или} \ |g(\overline{x})| \ < \ \varepsilon.
\end{equation}

\underline{Замечание}
\begin{enumerate}
    \item Метод Ньютона обладает высокой точностью и скоростью сходимости, если начальное приближение выбрано удачно. Если нет --- метод может разойтись. Тогда стоит выполнить несколько итераций другого метода, например, золотого сечения.
    \item \textbf{Модифицированный метод Ньютона.}
    
        Рассчетная схема метода:
        \begin{equation}
            \overline{x} \ = \ \overline{x}' \ - \ \frac{g(\overline{x}')}{g'(x_0)},
        \end{equation}
        где $x_0$ --- начальное приближение корня.

        В качестве очередного приближения для корня $x^*$ используется точка пересечения прямой, проходящей через $x_n$ и параллельной касательной, с осью $Ox$ в точке $x_0$.

        В данном методе меньше вычислений в рамках одной итерации, но самих итераций больше.
    \item \textbf{О вычислениии производных.}
        
        Конечно-разностная аппроксимация производных:
        \begin{equation}
            f'(\overline{x}) \ \approx \ \frac{f(\overline{x} \ + \ \delta) \ - \ f(\overline{x} \ - \ \delta)}{2 \delta}, \ \delta \ > \ 0;
        \end{equation}
        \begin{equation}
            f''(\overline{x}) \ \approx \ \frac{f(\overline{x} \ - \ \delta) \ - \ 2f(\overline{x}) \ + \ f(\overline{x} \ + \ \delta)}{\delta^2}, \ \delta \ > \ 0.
        \end{equation}
\end{enumerate}

Схема рассматриваемого метода представлена на рисунке \ref{img:algorithm}. 

\imgs{algorithm}{h!}{0.7}{Схема алгорима метода Ньютона}


\chapter{Практическая часть}

\begin{table}[H]
    \centering
	\caption{Результаты расчетов для задачи из индивидуального варианта}
    \label{tbl:task}
	\begin{tabular}{|c|c|c|c|c|}
        \hline
        № п/п & $\varepsilon$ & $N$ & $x^*$ & $f(x^*)$ \\ \hline
        1 & 0.01 & 8 & 0.4826521305 & -1.4738930849 \\ \hline
        2 & 0.0001 & 10 & 0.4824240968 & -1.4738932842 \\ \hline
        3 & 0.000001 & 14 & 0.4824180859 & -1.4738932844 \\ \hline
    \end{tabular}
\end{table}

\begin{table}[H]
    \centering
	\caption{Обобщающая таблица (для $\varepsilon = 1e-6$)}
    \label{tbl:task}
	\begin{tabular}{|c|l|c|c|c|}
        \hline
        № п/п & Метод & $N$ & $x^*$ & $f(x^*)$ \\ \hline
        1 & Поразрядного поиска & 50 & 0.4824180603 & -1.4738932844 \\ \hline
        2 & Золотого сечения & 31 & 0.4824184653 & -1.4738932844 \\ \hline
        3 & Парабол & 11 & 0.4824179876 & -1.4738932844 \\ \hline
        4 & Ньютона модифицированный & 14 & 0.4824180859 & -1.4738932844 \\ \hline
        5 & Функция \texttt{fminbnd} & 9 & 0.4824181903 & -1.4738932844 \\ \hline
    \end{tabular}
\end{table}

В листинге \ref{lst:lab_04} представлен код программы.

\mylisting[matlab]{lab_04.m}{firstline=1,lastline=52}{Код программы}{lab_04}{}

\end{document}
