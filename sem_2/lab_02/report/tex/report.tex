\documentclass{bmstu}

\begin{document}

\makereporttitle
{Информатика и системы управления (ИУ)}
{Программное обеспечение ЭВМ и информационные технологии (ИУ7)}
{\textbf{2}}
{Методы вычислений}
{Метод золотого сечения}
{6}
{ИУ7-22М}
{К.Э. Ковалец}
{П.А. Власов}


\setcounter{page}{2}
% \renewcommand{\contentsname}{Содержание} 
% \tableofcontents


\chapter{Теоретическая часть}

\textbf{Цель работы:} изучение метода золотого сечения для решения задачи одномерной минимизации.

\textbf{Задание:}
\begin{enumerate}
    \item Реализовать метод золотого сечения в виде программы на ЭВМ.
    \item Провести решение задачи:
    \begin{equation}
        \begin{cases}
            f(x) \rightarrow min, \\
            x \in [a, b]
        \end{cases}
    \end{equation}
    для данных индивидуального варианта;
    \item организовать вывод на экран графика целевой функции, найденной точки минимума $(x^*, \ f(x^*))$ и последовательности точек $(x_i, \ f(x_i))$, приближающих точку искомого минимума (для последовательности точек следует предусмотреть возможность <<отключения>> вывода ее на экран).
\end{enumerate}

\begin{table}[H]
    \centering
	\caption{Данные индивидуального варианта}
    \label{tbl:task}
	\begin{tabular}{|c|c|c|}
        \hline
        \textbf{№ вар.} & \textbf{Целевая функция $f(x)$} & \textbf{[a, b]} \\ \hline
        6 &
        \begin{minipage}[t]{12cm}\centering 
            \text{ch $(\frac{3x^3 \ + \ 2x^2 \ - \ 4x \ + \ 5}{3}) \ + \ $th $(\frac{x^3 \ - \ 3\sqrt{2}x \ - \ 2}{2x \ + \ \sqrt{2}}) \ - \ {2.5}$}
        \end{minipage} & 
        [0, 1] \\ \hline
    \end{tabular}
\end{table}

\clearpage

\section{Краткое описание метода золотого сечения}

Пусть $x_1, \ x_2 \in [a, \ b]$ --- пробные точки, которые расположены симметрично относительно середины отрезка. С целью уменьшения количества вычисляемых значений функции $f$ надо подобрать $x_1$ и $x_2$ так, чтобы при переходе к новому отрезку $[a_1, \ b_1] \subset [a, \ b]$ одна из них стала новой пробной точкой.

\begin{enumerate}
    \item Каждая из точек $x_1$ и $x_2$, используемых в данном методе, делит отрезок $[a, \ b]$ на 2 неравные части так, что 
    \begin{equation}
        \frac{\text{длина большей части отрезка}}{\text{длина всего отрезка}} \ = \ 
        \frac{\text{длина меньшей части отрезка}}{\text{длина большей части отрезка}}.
    \end{equation}
    Точки, обладающие этим свойством, называются точками золотого сечения отрезка $[a, \ b]$.
    \item На каждой итерации длина отрезка уменьшается в $\tau = \frac{\sqrt{5} - 1}{2}$ раз, поэтому после $n$ итераций длина соответствующего отрезка равна $\tau^n(b - a)$.
    \item Число $n$ итераций, необходимых для достижения заданной точности, составляет
    \begin{equation}
        2.1 \cdot ln(\frac{b - a}{2 \varepsilon}).
    \end{equation}
    \item Для выполнения первой итерации метода необходимо вычислить 2 значения функции $f$. Для выполнения остальных итерации необходимо вычислить 1 значения функции $f$. Таким образом, для выполнения $n$ итераций необходимо вычисление $N + 1$ значения функции. Точность $\varepsilon(N)$, которую можно обеспечить путем вычисления $N$ значений функции $f$, равна
    \begin{equation}
        \frac{1}{2} \tau^{N - 1} (b - a) \ \approx \ \tau^{N - 2} (b - a).
    \end{equation}
\end{enumerate}

\clearpage

Схема рассматриваемого метода представлена на рисунке \ref{img:algorithm}. 

\imgs{algorithm}{h!}{0.6}{Схема алгорима метода поразрядного поиска}


\chapter{Практическая часть}

\begin{table}[H]
    \centering
	\caption{Результаты расчетов для задачи из индивидуального варианта}
    \label{tbl:task}
	\begin{tabular}{|c|c|c|c|c|}
        \hline
        № п/п & $\varepsilon$ & $N$ & $x^*$ & $f(x^*)$ \\ \hline
        1 & 0.01 & 11 & 0.4787137637 & -1.4738433053 \\ \hline
        2 & 0.0001 & 20 & 0.4824455579 & -1.4738932816 \\ \hline
        3 & 0.000001 & 30 & 0.4824184653 & -1.4738932844 \\ \hline
    \end{tabular}
\end{table}

В листинге \ref{lst:lab_02} представлен код программы.

\mylisting[matlab]{lab_02.m}{firstline=1,lastline=67}{Код программы}{lab_02}{}

\end{document}
